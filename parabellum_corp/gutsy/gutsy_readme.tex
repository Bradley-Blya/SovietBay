\documentclass[a4paper,12pt]{article}
\RequirePackage{cmap}
\RequirePackage[utf8]{inputenc}
\RequirePackage[T2A]{fontenc}
\RequirePackage[english,russian]{babel}
\setcounter{tocdepth}{4}
\righthyphenmin=2
\tolerance=2000
\pdfcompresslevel=9
\begin{document}
\section{In gamey hawta}
Все спрайтосы и картинки обновлять только внутри процедуры \begin{verbatim}update_icons!\end{verbatim}
Система ввода-вывода \begin{verbatim}gutsy_IO\end{verbatim}
\section{Read me foken plebz}
\section{Haw to codenk}
Хочешь возиться в своей маленькой луже, взяв за основу оригинал Parabellum Corp.? Возись и не еби мозги своими коммитами. Хочешь помочь в разработке самого оригинала? Тогда запомни несколько правил.
\begin{enumerate}
  \item Все переменные называй с маленькой буквы. В имени переменной два или больше слов? Тогда используй верблюда, не будь пиздой.
  \item Все подпрограммы называй с большой буквы. И да, ты правильно подумал, тут тоже нужно использовать верблюда.
  \item Как в основной программе подпрограммы названы, нас не ебет, не будем мы лезть в чужой монастырь.
\end{enumerate}
\end{document} 